\documentclass[twocolumn]{article}
\usepackage{graphicx} % needed for including graphics e.g. EPS, PS

%\usepackage{epstopdf}
%\DeclareGraphicsRule{.eps}{pdf}{.pdf}{`epstopdf #1}
%\pdfcompresslevel=9

\usepackage{lscape}
\usepackage{fancyvrb}
\usepackage{float}
\usepackage{pdflscape}
\usepackage{multirow}
\usepackage{chngpage}
\usepackage{hyperref}

\DeclareGraphicsExtensions{.pdf}
\long\def\comment#1{}

% uncomment if don't want page numbers
% \pagestyle{empty}

%set dimensions of columns, gap between columns, and paragraph indent 
\setlength{\textheight}{8.75in}
\setlength{\columnsep}{0.375in}
\setlength{\textwidth}{6.8in}
\setlength{\topmargin}{0.0625in}
\setlength{\headheight}{0.0in}
\setlength{\headsep}{0.0in}
\setlength{\oddsidemargin}{-.19in}
\setlength{\parindent}{0pt}
\setlength{\parskip}{0.12in}
\makeatletter
\def\@normalsize{\@setsize\normalsize{10pt}\xpt\@xpt
\abovedisplayskip 10pt plus2pt minus5pt\belowdisplayskip 
\abovedisplayskip \abovedisplayshortskip \z@ 
plus3pt\belowdisplayshortskip 6pt plus3pt 
minus3pt\let\@listi\@listI}

% NEED Superscript and subscript in text mode...
\newcommand{\super}[1]{\ensuremath{^{\textrm{#1}}}}
\newcommand{\sub}[1]{\ensuremath{_{\textrm{#1}}}}
\newcommand{\sth}[0]{\super{th}}
\newcommand{\sst}[0]{\super{st}}
\newcommand{\snd}[0]{\super{nd}}
\newcommand{\srd}[0]{\super{rd}}

% provides scientific notation.  Us it like this:
% 3.2\e{-10} m  will give you: 3.2�10-10 m.
\providecommand{\e}[1]{\ensuremath{\times 10^{#1}}}

%need an 11 pt font size for subsection and abstract headings 
\def\subsize{\@setsize\subsize{12pt}\xipt\@xipt}
%make section titles bold and 12 point, 2 blank lines before, 1 after
\def\section{\@startsection {section}{1}{\z@}{1.0ex plus
1ex minus .2ex}{.2ex plus .2ex}{\large\bf}}
%make subsection titles bold and 11 point, 1 blank line before, 1 after
\def\subsection{\@startsection 
   {subsection}{2}{\z@}{.2ex plus 1ex} {.2ex plus .2ex}{\subsize\bf}}

% >>>>>>>>>>>>>>>>>>>>>>>>>>>>  Make DRAFT watermark appear on every page <<<<<<<<<<<<<<<<<<<
% Comment out this whole section to remove DRAFT watermark
\usepackage{graphicx,type1cm,eso-pic,color}
\makeatletter
          \AddToShipoutPicture{%
            \setlength{\@tempdimb}{.5\paperwidth}%
            \setlength{\@tempdimc}{.5\paperheight}%
            \setlength{\unitlength}{1pt}%
            \put(\strip@pt\@tempdimb,\strip@pt\@tempdimc){%
        \makebox(0,0){\rotatebox{55}{\textcolor[gray]{0.85}%
        {\fontsize{5cm}{5cm}\selectfont{DRAFT}}}}%
            }%
        }
% >>>>>>>>>>>>>>>>>>>>>>>>>>>>  Make DRAFT watermark appear on every page <<<<<<<<<<<<<<<<<<<
\makeatother

\begin{document}

% if you don't want the date printed, uncomment the next line
%\date{}

% >>>>>>>>>>>>>>>>>>>>>>>  Put your title here <<<<<<<<<<<<<<<<<<<<<<<<
% make title bold and 14 pt font (Latex default is non-bold, 16pt)
\title{\Large {\bf An Example of A Latex Document}}

% >>>>>>>>>>>>>>>>>>>>>>> Author's Names, Thanks or Affliation <<<<<<<<
\author{C. Grandin
}

\maketitle
\thispagestyle{empty}

\subsection*{\centering Abstract}

{\em Keywords:
  Latex, git, knitr

  This document is an example of what Git, Latex, and the R package 'knitr' can do. The three can be used together to create a
  document and to track changes easliy.
}

\section*{Introduction}
{\it Git} version control and {\it Latex} typesetting software can be use together to collaborate on documents and track changes.

The length-weight relationship for Strait of Georgia Pacific Hake were determined by using AD Model builder software ({\it ADMB}) \cite{ADMB}
to fit a simple 2-parameter nonlinear model ({\it Program \ref{lwcode}}) to the length-weight data acquired from sampling of the
midwater trawls for each species.
The minimization took place on the squared difference of the predicted weights and the actual weights according to the formula:

\begin{equation}\label{EQnorm2}
z=\sum_{i}(wpred_i-w_i)^2
\end{equation}

where {\it wpred} are the predicted weights and {\it w} are the actual weights.
The data came from 11 trawls. A Monte Carlo Markov Chain (MCMC) with a chain length of 10,000
was executed for the models and convergence was good.  The length-weight relationship is defined as:

\begin{equation}\label{EQlengthweight}
 \omega=\alpha \ell^{\beta}
\end{equation}

%% begin.rcode, echo=FALSE
%  source('utilities.r')
%  source('getOutput.r')
%% end.rcode
The values of $\alpha$ and $\beta$ were found to be \rinline{lw[[2]][1,]} and \rinline{lw[[2]][2,]} respectively.
The corresponding equations are:

\begin{equation}\label{LWHake}
w = \rinline{lw[[2]][1,]} \ell^{\rinline{lw[[2]][2,]}}
\end{equation}

Length-at-age was also examined for Pacific Hake, in a similar fashion as the length-weight relationship.
The code for this growth model can be found in {\it Program \ref{lacode}}.
The minimization took place on the squared difference of the predicted lengths and the observed lengths according to the formula:
\begin{equation}\label{LAnorm2}
z=\sum_{i}(\ell pred_i-\ell_i)^2
\end{equation}

for the Von Bertalanffy relationship:

\begin{equation}\label{VonB}
L_t = L_\infty (1 - e^{(-kt)})
\end{equation}

where $L_\infty$ is the asymptotic maximum fork length, $k$ is the growth coefficient, and $L_t$ is the fork length at time t, the age in years.
For Pacific Hake, the values of $L_\infty$ and $k$ were found to be \rinline{vonb[[2]][1,]} and \rinline{vonb[[2]][2,]} respectively.

\clearpage
\begin{figure*}
%% begin.rcode, echo=FALSE, lengthWeightFigure, dev='cairo_pdf'
%  plotLW(lw)
%% end.rcode
\caption{Length-weight relationship for Pacific Hake}
\label{figLengthWeight}
\end{figure*}

\begin{figure*}
%% begin.rcode, echo=FALSE, normFig, dev='cairo_pdf'
%  getShadeExample(func={function(x) dnorm(x,0,0.25)},from=-1,to=1, lwd=2, col="red", yaxp=c(0,3,30))
%% end.rcode
\caption{Example of a normal distribution using getShade function}
\label{figNormalDist}
\end{figure*}

\begin{figure*}
%% begin.rcode, echo=FALSE, chiSqFigFigure, dev='cairo_pdf'
%  getShadeExample(func={function(x) dchisq(x,100,df=10)},
%                  from=0,
%                  to=20,
%                  lwd=2,
%                  col="black",
%                  shade=getShade("green",opacity=20),
%                  yaxp=c(0,3,30))
%% end.rcode
\caption{Example of a chi square distribution using getShade function}
\label{figChisquareDist}
\end{figure*}

\begin{thebibliography}{99}

\bibitem{ADMB} Otter Research,
{\it An Introduction to AD Model Builder for use in Nonlinear Modeling and Statistics,
Otter Research Ltd., Nanaimo, B.C., Canada, 1994.
}

\end{thebibliography}

\clearpage
\newfloat{program}{h}{l}
\floatname{program}{Program}
\begin{program}
  \VerbatimInput[baselinestretch=1,fontsize=\footnotesize,numbers=left]{lengthweight.tpl}
  \caption{ADMB \cite{ADMB} TPL code for length-weight relationship model.}
  \label{lwcode}
\end{program}

\clearpage
\newfloat{program}{h}{l}
\floatname{program}{Program}
\begin{program}
  \VerbatimInput[baselinestretch=1,fontsize=\footnotesize,numbers=left]{vonb.tpl}
  \caption{ADMB \cite{ADMB} TPL code for length-age growth model.}
  \label{lacode}
\end{program}

\end{document}

